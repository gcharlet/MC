\documentclass[a4paper]{article}

%% Language and font encodings
\usepackage[frenchb]{babel}
\usepackage[utf8x]{inputenc}
\usepackage[T1]{fontenc}
\usepackage{minted} %compiler avec la commande -shell-escape
\usepackage{graphicx}

%% Todo List
\usepackage{enumitem,amssymb}
\newlist{todolist}{itemize}{2}
\setlist[todolist]{label=$\square$}
\usepackage{pifont}
\newcommand{\cmark}{\ding{51}}%
\newcommand{\xmark}{\ding{55}}%
\newcommand{\done}{\rlap{$\square$}{\raisebox{2pt}{\large\hspace{1pt}\cmark}}%
\hspace{-2.5pt}}
\newcommand{\wontfix}{\rlap{$\square$}{\large\hspace{1pt}\xmark}}

%% Sets page size and margins
\usepackage[a4paper,top=3cm,bottom=2cm,left=3cm,right=3cm,marginparwidth=1.75cm]{geometry}
\setlength{\parskip}{.5em}

\newcommand{\HRule}{\rule{\linewidth}{0.5mm}}

%-------------------------------------------------------------------------------
% TITLE PAGE
%-------------------------------------------------------------------------------

\title
{
	\LARGE{Problème HAC}
	\HRule \\ [0.5cm]
	\LARGE \textbf{\uppercase{Complexité et Calculabilité}}
	\HRule \\ [0.5cm]
}

\author{Guillaume CHARLET \\ Kenji FONTAINE}

\begin{document}

\null  % Empty line
\nointerlineskip  % No skip for prev line
\vfill
\let\snewpage \newpage
\let\newpage \relax
\maketitle
\let \newpage \snewpage
\vfill
\break % page break

%-------------------------------------------------------------------------------
% Table of Contents
%-------------------------------------------------------------------------------

\tableofcontents
\newpage

%-------------------------------------------------------------------------------
% Introduction
%-------------------------------------------------------------------------------

\section{Description du problème}
Dans un graphe G, un arbre couvrant est un arbre consituté uniquement d'arêtes de G et contenant tous les sommets de G. Ces arbres sont enracinés; un sommet est distingué et appelé la racine de l'arbre. La profondeur d'un tel arbre est égale à la distance maximale d'un sommet à la racine. Le problème est défini comme tel : \\
\textbf{Entrée :} Un graphe non-orienté G et un entier k. \\
\textbf{Sortie :} Existe-t-il un arbre couvrant de G de hauteur k?  

%-------------------------------------------------------------------------------
% Partie 2
%-------------------------------------------------------------------------------

\section{Partie 2 : HAC est NP-difficile}

todo

%-------------------------------------------------------------------------------
% Partie 3
%-------------------------------------------------------------------------------

\section{Réduction de HAC vers SAT}

Exprimons les contraintes suivantes : \\
\begin{itemize}

\item Pour chaque sommet v ∈ V , il y a un unique entier h t.q. xv,h est vrai.
\item Il y a un unique sommet v t.q. d(v) = 0 (“v est la racine”).
\item Il y a au moins un sommet v t.q. d(v) = k.
\item Pour chaque sommet v, si d(v) > 0, alors il existe un sommet u tel que uv ∈ E et d(u) = d(v) − 1 (“le sommet u est un parent potentiel de v dans l’arbre”).
\end{itemize}

%-------------------------------------------------------------------------------
% Partie 3
%-------------------------------------------------------------------------------

\section{Partie 3 : Entrées-sorties synchrones}

todo

%-------------------------------------------------------------------------------
% Partie 4
%-------------------------------------------------------------------------------

\section{Partie 4 : Format des graphes}

todo

%-------------------------------------------------------------------------------
% Partie 5
%-------------------------------------------------------------------------------

\section{Partie 5 : Solveur SAT}

todo

%-------------------------------------------------------------------------------
% End
%-------------------------------------------------------------------------------

\end{document}
