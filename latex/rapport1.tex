\documentclass[a4paper]{article}

%% Language and font encodings
\usepackage[frenchb]{babel}
\usepackage[utf8x]{inputenc}
\usepackage[T1]{fontenc}
\usepackage{minted} %compiler avec la commande -shell-escape
\usepackage{amssymb}
\usepackage{graphicx}

%% Todo List
\usepackage{enumitem,amssymb}
\newlist{todolist}{itemize}{2}
\setlist[todolist]{label=$\square$}
\usepackage{pifont}
\newcommand{\cmark}{\ding{51}}%
\newcommand{\xmark}{\ding{55}}%
\newcommand{\done}{\rlap{$\square$}{\raisebox{2pt}{\large\hspace{1pt}\cmark}}%
\hspace{-2.5pt}}
\newcommand{\wontfix}{\rlap{$\square$}{\large\hspace{1pt}\xmark}}

%% Sets page size and margins
\usepackage[a4paper,top=3cm,bottom=2cm,left=3cm,right=3cm,marginparwidth=1.75cm]{geometry}
\setlength{\parskip}{.5em}

\newcommand{\HRule}{\rule{\linewidth}{0.5mm}}

%-------------------------------------------------------------------------------
% TITLE PAGE
%-------------------------------------------------------------------------------

\title
{
	\LARGE{Problème HAC}
	\HRule \\ [0.5cm]
	\LARGE \textbf{\uppercase{Complexité et Calculabilité}}
	\HRule \\ [0.5cm]
}

\author{Guillaume CHARLET \\ Kenji FONTAINE}

\begin{document}

\null  % Empty line
\nointerlineskip  % No skip for prev line
\vfill
\let\snewpage \newpage
\let\newpage \relax
\maketitle
\let \newpage \snewpage
\vfill
\break % page break

%-------------------------------------------------------------------------------
% Table of Contents
%-------------------------------------------------------------------------------

\tableofcontents
\newpage

%-------------------------------------------------------------------------------
% Introduction
%-------------------------------------------------------------------------------

\section{Description du problème}
Dans un graphe G, un arbre couvrant est un arbre consituté uniquement d'arêtes de G et contenant tous les sommets de G. Ces arbres sont enracinés; un sommet est distingué et appelé la racine de l'arbre. La profondeur d'un tel arbre est égale à la distance maximale d'un sommet à la racine. Le problème est défini comme tel : \\
\textbf{Entrée :} Un graphe non-orienté G et un entier k. \\
\textbf{Sortie :} Existe-t-il un arbre couvrant de G de hauteur k?

%-------------------------------------------------------------------------------
% Partie 2
%-------------------------------------------------------------------------------

\section{Partie 2 : HAC est NP-difficile}

todo

%-------------------------------------------------------------------------------
% Partie 3
%-------------------------------------------------------------------------------

\section{Réduction de HAC vers SAT}

Exprimons les contraintes suivantes : \\
\begin{enumerate}

\item Pour chaque sommet v ∈ V, il y a un unique entier h tq $x_{v,h}$ est vrai.
\item Il y a un unique sommet v tq d(v) = 0 (“v est la racine”).
\item Il y a au moins un sommet v tq d(v) = k.
\item Pour chaque sommet v, si d(v) > 0, alors il existe un sommet u tel que uv ∈ E et d(u) = d(v) − 1 (“le sommet u est un parent potentiel de v dans l’arbre”).
\end{enumerate}

\begin{enumerate}
%%%%% CONTRAINTE 1 %%%%%
\item
Pour cette contrainte, nous avons choisi de séparer le problème en deux.
Dans un premier temps, nous avons exprimé le fait qu'il y ait au moins un entier
h tq $x_{v, h}$ est vrai. Pour cela, nous appliquons un OU logique entre tous les
sommets du graphe G et ce, pour chaque valeur allant de 0 à k. Ceci se traduisant
par les clauses suivantes :
\begin{center}
$\bigwedge\limits_{v∈V}^{} \bigvee\limits_{i=0}^{k} x_{v,i}$
\end{center}

Puis nous avons exprimé le fait qu'il n'y ait au plus qu'un seul entier h tq
$x_{v, h}$ est vrai. Nous exprimons cela par un NON ET logique entre toutes les
paires de sommets différents du graphe G. En appliquant la loi de De Morgan, nous
arrivons aux clauses suivantes :
\begin{center}
$\bigwedge\limits_{v∈V}^{} \bigwedge\limits_{i=0}^{k} \bigwedge\limits_{j=i+1}^{k-1} \neg x_{v,i} \vee \neg x_{v,j} $
\end{center}

%%%%% CONTRAINTE 2 %%%%%
\item
De façon similaire à la contrainte précédente, nous avons séparé ce problème en
deux. Tout d'abord nous exprimons le fait qu'il y ait au moins une racine, puis
le fait qu'il n'y ait qu'une seule racine. Ce qui nous donne les clauses : \\
\begin{center}
$\bigvee\limits_{v∈V}^{} x_{v,0} $
\end{center}

\begin{center}
$\bigwedge\limits_{u!=v∈V}^{} \neg x_{u,0} \vee \neg x_{v,0}$
\end{center}







\end{enumerate}
%-------------------------------------------------------------------------------
% Partie 3
%-------------------------------------------------------------------------------

\section{Partie 3 : Entrées-sorties synchrones}

todo

%-------------------------------------------------------------------------------
% Partie 4
%-------------------------------------------------------------------------------

\section{Partie 4 : Format des graphes}

todo

%-------------------------------------------------------------------------------
% Partie 5
%-------------------------------------------------------------------------------

\section{Partie 5 : Solveur SAT}

todo

%-------------------------------------------------------------------------------
% End
%-------------------------------------------------------------------------------

\end{document}
