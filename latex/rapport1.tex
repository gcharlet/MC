\documentclass[a4paper]{article}

%% Language and font encodings
\usepackage[frenchb]{babel}
\usepackage[utf8x]{inputenc}
\usepackage[T1]{fontenc}
\usepackage{minted} %compiler avec la commande -shell-escape
\usepackage{amssymb}
\usepackage{graphicx}

%% Todo List
\usepackage{enumitem,amssymb}
\newlist{todolist}{itemize}{2}
\setlist[todolist]{label=$\square$}
\usepackage{pifont}
\newcommand{\cmark}{\ding{51}}%
\newcommand{\xmark}{\ding{55}}%
\newcommand{\done}{\rlap{$\square$}{\raisebox{2pt}{\large\hspace{1pt}\cmark}}%
\hspace{-2.5pt}}
\newcommand{\wontfix}{\rlap{$\square$}{\large\hspace{1pt}\xmark}}

%% Sets page size and margins
\usepackage[a4paper,top=3cm,bottom=2cm,left=3cm,right=3cm,marginparwidth=1.75cm]{geometry}
\setlength{\parskip}{.5em}

\newcommand{\HRule}{\rule{\linewidth}{0.5mm}}

%-------------------------------------------------------------------------------
% TITLE PAGE
%-------------------------------------------------------------------------------

\title
{
	\LARGE{Problème HAC}
	\HRule \\ [0.5cm]
	\LARGE \textbf{\uppercase{Complexité et Calculabilité}}
	\HRule \\ [0.5cm]
}

\author{Guillaume CHARLET \\ Kenji FONTAINE}

\begin{document}

\null  % Empty line
\nointerlineskip  % No skip for prev line
\vfill
\let\snewpage \newpage
\let\newpage \relax
\maketitle
\let \newpage \snewpage
\vfill
\break % page break

%-------------------------------------------------------------------------------
% Table of Contents
%-------------------------------------------------------------------------------

\tableofcontents
\newpage

%-------------------------------------------------------------------------------
% Introduction
%-------------------------------------------------------------------------------

\section{Description du problème}
Dans un graphe G, un arbre couvrant est un arbre constituté uniquement d'arêtes de G et contenant tous les sommets de G. Ces arbres sont enracinés; un sommet est distingué et appelé la racine de l'arbre. La profondeur d'un tel arbre est égale à la distance maximale d'un sommet à la racine. Le problème est défini comme tel : \\
\textbf{Entrée :} Un graphe non orienté G et un entier k. \\
\textbf{Sortie :} Existe-t-il un arbre couvrant de G de hauteur k?

%-------------------------------------------------------------------------------
% Partie 2
%-------------------------------------------------------------------------------

\section{Partie 2 : HAC est NP-difficile}

\begin{enumerate}

\item %%% question 1 %%%

Montrons qu'il existe une réduction polynomiale du problème Chemin hamiltonien
vers HAC. Les données d'entrée sont les mêmes, chaque problème prenant en entrée
un graphe G. Il nous faut donc copier les données d'un problème à l'autre, se
faisant en temps linéaire. Le problème HAC requiert un entier supplémentaire,
nous coûtant un temps logarithmique.\\
Il nous faut donc maintenant montrer que toute instance positive (respectivement négatives)
 du Chemin hamiltonien soit également positive (respectivement négative) chez HAC.
Si un graphe admet un chemin hamiltonien, cela signifie que le graphe est connexe.
De plus, nous pouvons en partant d'un sommet parcourir l'ensemble des sommets du
graphe, en ne passant qu'une seule fois par chaque sommet. Ce qui nous permettrait
de créer un arbre "ligne" de la même taille que celle du chemin hamiltonien. Et
si un graphe n'admet pas de chemin hamiltonien, cela signifie que le graphe n'est
pas connexe. Nous ne pouvons pas extraire un arbre couvrant d'un tel graphe.

\item %%% question 2 %%%

Nous avons vu en cours que le chemin hamiltonien appartenait à la classe NP complet
et nous avons montré qu'il existait une réduction en temps polynomiale d'hamiltonien
vers HAC. Le problème HAC est donc lui aussi un problème NP complet.

\end{enumerate}
%-------------------------------------------------------------------------------
% Partie 3
%-------------------------------------------------------------------------------

\newpage
\section{Réduction de HAC vers SAT}

Exprimons les contraintes suivantes :\\
\begin{enumerate}

\item Pour chaque sommet v ∈ V, il y a un unique entier h tq $x_{v,h}$ est vrai.
\item Il y a un unique sommet v tq d(v) = 0 (“v est la racine”).
\item Il y a au moins un sommet v tq d(v) = k.
\item Pour chaque sommet v, si d(v) > 0, alors il existe un sommet u tel que uv ∈ E et d(u) = d(v) − 1 (“le sommet u est un parent potentiel de v dans l’arbre”).
\end{enumerate}
\vspace{0.5cm}

\begin{enumerate}
%%%%% CONTRAINTE 1 %%%%%
\item
Pour cette contrainte, nous avons choisi de séparer le problème en deux.
Dans un premier temps, nous avons exprimé le fait qu'il y ait au moins un entier
h tq $x_{v, h}$ est vrai. Pour cela, nous appliquons un OU logique entre tous les
sommets du graphe G et ce, pour chaque valeur allant de 0 à k. Ceci se traduisant
par les clauses suivantes :
\begin{center}
$\bigwedge\limits_{v∈V}^{} \bigvee\limits_{i=0}^{k} x_{v,i}$
\end{center}

Puis nous avons exprimé le fait qu'il n'y ait au plus qu'un seul entier h tq
$x_{v, h}$ est vrai. Nous exprimons cela par un NON ET logique entre toutes les
paires de sommets différents du graphe G. En appliquant la loi de De Morgan, nous
arrivons aux clauses suivantes :
\begin{center}
$\bigwedge\limits_{v∈V}^{} \bigwedge\limits_{i=0}^{k} \bigwedge\limits_{j=i+1}^{k-1} \neg x_{v,i} \vee \neg x_{v,j} $
\end{center}

%%%%% CONTRAINTE 2 %%%%%
\item
De façon similaire à la contrainte précédente, nous avons séparé ce problème en
deux. Tout d'abord nous exprimons le fait qu'il y ait au moins une racine, puis
le fait qu'il n'y ait qu'une seule racine. Ce qui nous donne les clauses : \\
\begin{center}
$\bigvee\limits_{v∈V}^{} x_{v,0} $
\end{center}

\begin{center}
$\bigwedge\limits_{u!=v∈V}^{} \neg x_{u,0} \vee \neg x_{v,0}$
\end{center}

%%%%% CONTRAINTE 3 %%%%%
\item
Cette contrainte se traduit trivialement en clause, il nous suffit d'appliquer
un OU logique entre chacun des sommets du graphe pour une valeur donnée k.
Soit : \\
\begin{center}
$\bigvee\limits_{v∈V}^{} x_{v,k} $
\end{center}

%%%%% CONTRAINTE 4 %%%%%
\item
Pour cette contrainte, nous voulons que si d(v) > 0, une certaine propriété soit
satisfaite. Autrement dit, si $x_{v, i}$ est faux, nous voulons que la clause
soit vérifiée puisque celle-ci n'a pas de sens concret. Pour cela nous utilisons
la négation de cette variable associée à un OU logique. De cette sorte, lorsque
la variable est vraie, et donc que la condition d(v) > 0 est vérifiée, nous devons
considérer la partie droite du OU. \\
La seconde partie du OU logique représente l'existence d'un sommet u comme un
parent potentiel de v dans le graphe. Autrement dit, notre OU logique sera
vérifié lorsque $x_{u, i-1}$ est vraie.

Nous amenant ainsi aux clauses suivantes : \\
\begin{center}
$\bigwedge\limits_{u,v∈V}^{} \bigwedge\limits_{i=1}^{k} \neg x_{v,i} \vee x_{u,i-1}$
\end{center}

\end{enumerate}

%-------------------------------------------------------------------------------
% End
%-------------------------------------------------------------------------------

\end{document}
